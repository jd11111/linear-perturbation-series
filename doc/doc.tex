\documentclass[11pt, a4paper]{article} % first option = text size, second = paper size, choose "article" for no chapters etc., else choose "report", only use "book" for irl printed book
\usepackage[utf8]{inputenc} % better encoding of input
\usepackage[T1]{fontenc} % better encoding of font
\usepackage{lmodern} % better font
\usepackage[english]{babel} %localization of document
\usepackage{amssymb} %special math symbols + math fonts like frak cal etc...
\usepackage{mathtools} %math env stuff (no need for amsmath)
\usepackage{mathrsfs} % script math letters (\mathscr{})
\usepackage{amsthm} % theorem, proposition env etc.
\usepackage{hyperref} %links and pdf meta-data in document
\usepackage{caption} %captions
\usepackage{graphicx} %figures etc
\usepackage{enumerate} %better enumerate i.e. different labeling (latin numerals, letters etc.)
%\usepackage{subfiles}%enable multifile project
\usepackage{siunitx} %package to display physics units, uncertainties etc.
\usepackage[version=4]{mhchem} %chemistry package i.e. to write molecule formulas
\usepackage{tikz} %tikz can be used to construct figures in tex
\usetikzlibrary{cd} %tikz libary for commutative diagrams
\usepackage{csquotes} %better quotation enviroment
\usepackage[backend=biber, style=numeric,sorting=none]{biblatex}%bibliography as mentioned in text
\usepackage{booktabs} %better tables

%SETTINGS:
\sisetup{separate-uncertainty = true} %display uncertainties with +/-
\numberwithin{equation}{subsection} %equation numbering (different counter for every level)
\addbibresource{sources.bib} %add the bibliography file
\hypersetup{%settings for hyperref
    colorlinks=true,
    linkcolor=blue,
    filecolor=magenta,
    urlcolor=cyan,
    pdftitle={}
}

%the following block makes the optional text in amsthm envs bold:
\makeatletter
\def\th@plain{%
  \thm@notefont{}% same as heading font
  \itshape % body font
}
\def\th@definition{%
  \thm@notefont{}% same as heading font
  \normalfont % body font
}
\makeatother

\setcounter{MaxMatrixCols}{20}%increase max number of matrix columns

%amsthm settings
\theoremstyle{definition} %upright text
\newtheorem{definition}{Definition}[section] %two part counter (first incremented at each section)
\newtheorem{example}[definition]{Example} % all subsequent envs share counter with defi
\newtheorem{remark}[definition]{Remark}
\theoremstyle{plain} %italics
\newtheorem{lemma}[definition]{Lemma}
\newtheorem{proposition}[definition]{Proposition}

%definition of math operators
\DeclareMathOperator{\ran}{ran}
\DeclareMathOperator{\lin}{lin}
\DeclareMathOperator{\rad}{Rad}
\DeclareMathOperator{\tr}{tr}
\DeclareMathOperator{\obj}{Obj}
\DeclareMathOperator{\mor}{Mor}
\DeclareMathOperator{\ev}{eval}
\DeclareMathOperator{\re}{Re}
\DeclareMathOperator{\essran}{essRan}
\DeclareMathOperator{\im}{Im}
\DeclareMathOperator{\dom}{dom}
\newcommand{\catname}[1]{\mathbf{#1}}

\title{Linear Perturbation Series}
\author{Jannik Daun}
\begin{document}
\maketitle
\begin{abstract}
	The perturbation series for the eigenvalues of a linear perturbation
	(Rayleigh-Schrödinger series) is derived in the context of bounded operators on a Banach space.
	The Haskell code implementation to compute the coefficients in the linear perturbation series is briefly described.
\end{abstract}


\section{Introduction to the Main Result}
Let $X$ be a complex Banach space.
For example $X= \mathbb{C}^n$.
Denote by $\mathscr{B}(X)$ the space of all bounded linear operators $X \to X$.
So $\mathscr{B} (X) \cong \mathbb{C}^{n \times n}$ in the case $X= \mathbb{C}^n$.
Let $T_0 , \ T_1 \in \mathscr{B}(X)$.
Define $T : \mathbb{C} \to \mathscr{B}(X)$ by $T (x) = T_0 + x \cdot T_1$.
Then $T$ is called a linear perturbation of $T_0$ by $T_1$.
Let $\lambda_0$ be a semi-simple eigenvalue of $T_0$ that is isolated in $T_0$'s spectrum.
Let $m< \infty$ be the dimension of the eigenspace of $T_0$ to eigenvalue $\lambda_0$.

Then there exists a sequence $(\alpha_n)_{n \in \mathbb{N}}$ of complex numbers such that the following is true:
The sum $\alpha (x)$
of the eigenvalues of $T(x)$, where each eigenvalue is counted according to the dimension of the corresponding
generalized eigenspace, is given by
\begin{equation}
	\boxed{\alpha (x) = m \cdot \lambda_0 + \sum_{n=1}^\infty \alpha_n x^n}
\end{equation}
for $x$ in some disk $D \subset \mathbb{C}$ centered at $0$. The most interesting case is of course when $m=1$. Then $\alpha(x)$ is an eigenvalue of $T(x)$ with $\lim_{x \to 0}\alpha(x) =\lambda_0$.

Let $n \in \mathbb{N}$.
Then the coefficient $\alpha_n$ can be obtained as follows:
Define
\begin{equation}
	S_n := \{ (j_1, \dots, j_n ) \in \mathbb{N}^n : j_1 + \cdots + j_n = n-1 \}.
\end{equation}
Then the cyclic group of $n$ elements, $C_n$, acts on $S_n$ by circularly shifting the index of the tuple.
Let $O_n \subset S_n$ be such that $O_n$ contains exactly one element of every orbit of the action of $C_n$ on $S_n$.
Then
\begin{equation}
	\boxed{
		\alpha_n = \sum_{o \in O_n }
		\tr ( T_{1} \cdot  V_{o_1-1} \cdots T_{1} \cdot V_{o_n-1}),}
\end{equation}
where $V_{j}$ is the $j$-th coefficient in the Laurent expansion of the resolvent of $T_0$
around $\lambda_0$.

The code computes $\alpha_n$ in the case $X = \mathbb{C}^n$.
The function pertCoeff in app/PerturbationSeries.hs returns a list of the coefficients $\alpha_n$ up to a given order.
The algorithm to compute $O_n$ (as a list of lists instead of a set of tuples) is implemented in app/Combinatorics.hs.
for mathematical and implementation details on the computation of $O_n$ see my blog post \cite{blog}.

\section{Perturbation Series}
In this section the main result is derived.
The main source is Katos book \cite{kato_perturbation}.
\subsection{Laurent-Expansion of the Resolvent}
Let $X$ be a complex Banach space and $T_0 :X \to X$ a bounded linear
operator and $R$ the resolvent of $T_0$.
Let $ \lambda_0 \in \sigma (T_0)$ be an isolated point of the spectrum
of $T_0$. Let $P$ be the spectral projection associated to $\lambda_0$
and $Q:=I-P$ the complementary projection.
Then the Laurent-expansion of $R$ around $\lambda_0$ is:
\begin{equation}
	\label{eq:resolvent-laurent-expansion}
	\begin{split}
		R(\lambda) &=
		\sum_{n=1}^\infty (\lambda - \lambda_0)^{-n} \cdot (T_0- \lambda_0I)^{n-1} P  \\
		&+ \sum_{n=0}^\infty (\lambda - \lambda_0)^n \cdot (-1)^n \cdot  S^{n+1} Q ,
	\end{split}
\end{equation}
where $S$ is the inverse of $(\lambda_0 I- T_0)|_{Q(X)}: Q(X) \to Q(X)$.
$S$ is called the reduced resolvent of $T_0$ at $\lambda_0$.
The inner radius of convergence of the Laurent-series is $0$ and the outer $d(\lambda_0 , \sigma(T))$.

\subsection{Perturbation-series for the Resolvent}
Let $(T_n)_{n \in \mathbb{N}}$ be a sequence in $\mathscr{B}(X)$ (bounded  linear operators on $X$).
Let $T : D \to \mathscr{B}(X)$ defined by
\begin{equation}
	T(x) := \sum_{n=0}^\infty T_n x^n,
\end{equation}
where $D$ is the open disk centered at 0 whose radius is the radius of
convergence of the power series (assumed $>0$).

Let $U:= \{ (\lambda, x) \in \mathbb{C}^2 : \lambda I - T(x) \ \text{invertible} \}$
and define
$R: U \to \mathscr{B}(X)$ by
\begin{equation}
	R(\lambda, x) := (\lambda I - T(x))^{-1}.
\end{equation}
Let $A(x):= T(x)-T_1$.
Now
\begin{equation}
	\lambda I - T (x) = \lambda I - T_0 - A(x) = ( I - A(x) R(\lambda,0)) (\lambda I -T_0)
\end{equation}
and so by using the geometric series:
\begin{equation}
	\label{eq:eq_R_expansion}
	R(\lambda,x) = R(\lambda,0)  ( I - A(x) R(\lambda,0))^{-1} =
	R(\lambda,0)  \sum_{n=0}^\infty  (A(x) R(\lambda,0) \big)^n
\end{equation}
with a $>0$ radius of convergence.
From this it follows, by collecting the terms that have the same power of $x$,
that $R(\lambda,x)$ is analytic in $x$ at $0$.
Similarly one can show that $R$ is bi-analytic on $U$ as well.
\subsection{Eigenvalue Perturbation-series}
Let $\lambda_0$ be an isolated element of $\sigma (T_0)$ with finite dimensional
generalized eigenspace (which is equivalent to the Laurent-expansion having only finitely many non-zero coefficients to negative power).
Let $m$ be the dimension of the generalized eigenspace.
Let $ \gamma$ be a circular path surrounding $\lambda_0$ once in the positiv sense.
It follows from abstract properties of the holomorphic functional calculus,
that for $x \in \mathbb{C}$ "small enough"
\begin{equation}
	P(x) = \frac{1}{2\pi i} \int_\gamma  R(\lambda,x) d \lambda
\end{equation}
is the projection onto sum of all the generalized eigenspaces to the eigenvalues
that have split from $\lambda_0$ ("split eigenvalues").
Therefore
\begin{equation}
	\alpha (x) := \tr T(x) P(x)
\end{equation}
is the weighted (by the dimension of the generalized eigenspaces) sum of all
the split eigenvalues.

Therefore
\begin{equation}
	\alpha (x) -  m \lambda_0 = \tr (T(x) - \lambda_0 I ) P(x)
	= \frac{1}{2\pi i} \tr \int_\gamma (\lambda - \lambda_0 ) R(\lambda, x) d\lambda
\end{equation}
Inserting the expansion for $R$ (equation \ref{eq:eq_R_expansion}):
\begin{equation}
	\begin{split}
		&\alpha (x) -  m \lambda_0
		= \frac{1}{2\pi i} \tr \int_\gamma (\lambda - \lambda_0 )  R(\lambda,0)  \sum_{n=1}^\infty  (A(x) R(\lambda,0) \big)^n d\lambda \\
		&= -\frac{1}{2\pi i} \tr \int_\gamma (\lambda - \lambda_0) \sum_{n=1}^\infty \frac{1}{n}
		\bigg(\frac{\partial }{ \partial z}  (A(x) R(z,0))^n \bigg) (\lambda) d \lambda  \\
		&= \frac{1}{2\pi i} \tr \int_\gamma  \sum_{n=1}^\infty \frac{1}{n}
		(A(x) R(\lambda,0))^n  d \lambda.
	\end{split}
\end{equation}
In the first equality the $n=0$ term vanishes.
For the second the cyclicity of the trace and the derivative of the resolvent are used.
For the third partial integration is used.
Collecting powers of $x$ in the above:
\begin{equation}
	\alpha (x) - m \lambda_0 = \sum_{n=1}^\infty \alpha_n x^n
\end{equation}
with
\begin{equation}
	\alpha_n := \sum_{k=1}^n  \sum_{ \substack{(i_1, \dots, i_k) \in (\mathbb{N} \setminus \{0\})^k \\ i_1 + \cdots + i_k = n}}
	\frac{1}{2 \pi k i} \tr \int_\gamma T_{i_1} R(\lambda, 0)  \cdots  T_{i_k} R(\lambda, 0) d \lambda .
\end{equation}
For $n \in \mathbb{Z}$ let $V_n \in \mathscr{B}(X)$ be the coefficient of $(\lambda-\lambda_0)^{n}$ in the Laurent expansion of $T_0$
around $\lambda_0$, then
using the residue theorem and the Laurent expansion of the Resolvent (equation \ref{eq:resolvent-laurent-expansion}) around $\lambda_0$ to evaluate the integral:
\begin{equation}
	\alpha_n =\sum_{k=1}^n
	\sum_{ \substack{(i_1, \dots, i_k) \in (\mathbb{N} \setminus \{0\})^k \\ i_1 + \cdots + i_k = n}}
	\frac{1}{k} \sum_{\substack{(j_1, \dots, j_k) \in \mathbb{Z}^k\\ j_1 + \cdots + j_k = -1 }}
	\tr T_{i_1} V_{j_1} \cdots T_{i_k} V_{j_k}.
\end{equation}
\subsection{Eigenvalue Perturbation-series for Linear Perturbation and Semi-simple Eigenvalue}
In the special case of a linear perturbation ($T_n =0$ for all $n \in \mathbb{N}$ with $n\geq2$)
and a semi simple eigenvalue (meaning that $(T_0- \lambda_0 I) P =0$)
we obtain
\begin{equation}
	\begin{split}
		\alpha_n &=  \frac{1}{n} \sum_{ \substack{(j_1, \dots, j_n) \in \mathbb{Z}^n \\ j_1 + \cdots + j_n = -1}}
		\tr T_{1} V_{j_1} \cdots T_{1} V_{j_n}\\
		&= \frac{1}{n} \sum_{ \substack{(j_1, \dots, j_n) \in \mathbb{N}^n \\ j_1 + \cdots + j_n = n -1}}
		\tr T_{1} V_{j_1-1} \cdots T_{1} V_{j_n-1}.
	\end{split}
\end{equation}
To reduce the computational complexity we want to find
all $(j_1, \dots, j_n) \in \mathbb{N}^n  $ with $ j_1 + \cdots + j_n = n -1$
that are the same
up to cyclic permutations.
It turns out, that every orbit under the cyclic group action on these indices has $n$ elements.
For $n \in \mathbb{N}$ let $O_n$ be a set of indices so that each element is a representant of each distinct orbit of the cyclic group acting on the the indices
$(j_1, \dots, j_n) \in \mathbb{N}^n  $ with $ j_1 + \cdots + j_n = n -1$.
Then
\begin{equation}
	\begin{split}
		\alpha_n &= \sum_{o \in O_n }
		\tr T_{1} V_{o_1-1} \cdots T_{1} V_{o_n-1}.
	\end{split}
\end{equation}
\section{Exact Solution of $2 \times 2$ Case}
In this section the linear perturbations series for a special toy case will be solved by hand.
This allows to compare the results generated by the code to the analytic result.
Namely the coefficients of the powers of $x$ in equation \ref{eq:Eanalytical}
can be compared to the ones calculated using the code.
This comparison is implemented in app/test.hs.
For $E_1, E_2, a \in \mathbb{R}$
let
$$H:=
	\begin{pmatrix}
		E_1 & 0   \\
		0   & E_2
	\end{pmatrix}$$
and
$$
	V :=
	\begin{pmatrix}
		0 & a \\
		a & 0
	\end{pmatrix}.
$$
Define $T: \mathbb{C} \to \mathbb{C}^{2 \times 2}$
by $T \ x =  H + x \cdot V$.
Assume that $E_1 \neq E_2$, then
the two eigenvalues $E_\pm$ of $T \  x$ (the roots of the characteristic polynomial) can be found as
\begin{equation}
	E_\pm (x) =  \frac{E_{1}+ E_2}{2}
	\pm \frac{|E_1 -E_2|}{2} \sqrt{
		1 + \bigg(\frac{2 a x}{E_1-E_2}\bigg)^2
	}.
\end{equation}
Now upon potentially relabeling $E_\pm$ we obtain
\begin{equation}
	E_\pm (x) =  \frac{E_{1}+ E_2}{2}
	\pm \frac{E_1 -E_2}{2} \sqrt{
		1 + \bigg(\frac{2 a x}{E_1-E_2}\bigg)^2
	}.
\end{equation}
Now the binomial series says that for $x \in \mathbb{C}$ with $|x|<1$:
\begin{equation}
	(1+x)^{1/2} = \sum_{k=0}^\infty
	{1/2 \choose k} x^k,
\end{equation}
where the radius of convergence of the power series is 1.
Therefore
\begin{equation}
	\label{eq:Eanalytical}
	\begin{split}
		E_\pm (x) &=  \frac{E_{1}+ E_2}{2}
		\pm \frac{E_1 -E_2}{2} \sum_{k=0}^\infty
		{1/2 \choose k} \bigg(\frac{2 a x}{E_1-E_2}\bigg)^{2k} \\
		&= E_{1/2} \pm \frac{1}{2} \sum_{k=1}^\infty
		{1/2 \choose k} \frac{(2 a)^{2k}}{(E_1-E_2)^{2k-1}} x^{2k}
	\end{split}
\end{equation}
where the radius of convergence $r$ of the power series is
\begin{equation}
	r = \frac{|E_1 -E_2|}{2 |a|}.
\end{equation}
\printbibliography %print the bibliography
\end{document}
